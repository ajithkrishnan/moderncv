%%%%%%%%%%%%%%%%%%%%%%%%%%%%%%%%%%%%%%%%%
% "ModernCV" CV and Cover Letter
% LaTeX Template
% Version 1.3 (29/10/16)
%
% This template has been downloaded from:
% http://www.LaTeXTemplates.com
%
% Original author:
% Xavier Danaux (xdanaux@gmail.com) with modifications by:
% Vel (vel@latextemplates.com)
%
% License:
% CC BY-NC-SA 3.0 (http://creativecommons.org/licenses/by-nc-sa/3.0/)
%
% Important note:
% This template requires the moderncv.cls and .sty files to be in the same 
% directory as this .tex file. These files provide the resume style and themes 
% used for structuring the document.
%
%%%%%%%%%%%%%%%%%%%%%%%%%%%%%%%%%%%%%%%%%

%----------------------------------------------------------------------------------------
%	PACKAGES AND OTHER DOCUMENT CONFIGURATIONS
%----------------------------------------------------------------------------------------

\documentclass[11pt,a4paper,roman]{moderncv} % Font sizes: 10, 11, or 12; paper sizes: a4paper, letterpaper, a5paper, legalpaper, executivepaper or landscape; font families: sans or roman

\moderncvstyle{classic} % CV theme - options include: 'casual' (default), 'classic', 'oldstyle' and 'banking'
\moderncvcolor{blue} % CV color - options include: 'blue' (default), 'orange', 'green', 'red', 'purple', 'grey' and 'black'

\usepackage{lipsum} % Used for inserting dummy 'Lorem ipsum' text into the template

\usepackage[scale=0.75]{geometry} % Reduce document margins
%\setlength{\hintscolumnwidth}{3cm} % Uncomment to change the width of the dates column
%\setlength{\makecvtitlenamewidth}{10cm} % For the 'classic' style, uncomment to adjust the width of the space allocated to your name

% to comment out blocks 
\usepackage{verbatim}

% to center align and specify column width in tables
\usepackage{array}

% german Language, new writing rules
\usepackage[T1]{fontenc}
\usepackage[utf8]{inputenc}
\usepackage[ngerman]{babel}  %<========== Language German, new orthography

\usepackage{fontawesome}

%----------------------------------------------------------------------------------------
%	NAME AND CONTACT INFORMATION SECTION
%----------------------------------------------------------------------------------------

%\firstname{\Huge{\scshape{Ajith}}} % Your first name
%\familyname{\scshape{Krishnan}} % Your last name

\name{\scshape{Ajith}}{\scshape{Krishnan}}

% All information in this block is optional, comment out any lines you don't need
\title{\Large{Curriculum Vitae}}
%\phone{(000) 111 1112}
%\fax{(000) 111 1113}
\email{ajith.krishnan@alumni.fh-aachen.de}
%\homepage{www.linkedin.com/in/ajith-krishnan}{linkedin.com/ajith-krishnan} % The first argument is the url for the clickable link, the second argument is the url displayed in the template - this allows special characters to be displayed such as the tilde in this example
% \homepage{www.github.com/ajithkrishnan}{staff.org.edu/$\sim$jsmith} % The first argument is the url for the clickable link, the second argument is the url displayed in the template - this allows special characters to be displayed such as the tilde in this example
\social[linkedin]{linkedin.com/in/ajith-krishnan}
\social[github]{github.com/ajithkrishnan} 
\extrainfo{DOB: 04.01.1993 | Nationality: Indian}
%\extrainfo{\httplink{github.com/ajithkrishnan}}
%\address{Scheibenstraße 14}{Aachen, Deutschland 52070}
\mobile{+49 15758660972}
\photo[89pt][0.3pt]{../pictures/bild.jpg} % The first bracket is the picture height, the second is the thickness of the frame around the picture (0pt for no frame)
%\quote{"A witty and playful quotation" - John Smith}

%----------------------------------------------------------------------------------------

\begin{document}


%\begin{comment}
%----------------------------------------------------------------------------------------
%	COVER LETTER
%----------------------------------------------------------------------------------------

% To remove the cover letter, comment out this entire block

\clearpage

\recipient{HR Department}{Corporation\\123 Pleasant Lane\\12345 City, State} % Letter recipient
\date{\today} % Letter date
\opening{Dear Sir or Madam,} % Opening greeting
\closing{Sincerely yours,} % Closing phrase
\enclosure[Attached]{curriculum vit\ae{}} % List of enclosed documents

\makelettertitle % Print letter title

As per our conversation, last Tuesday, I am sending you my CV and github/gitlab links to some projects i have worked on. 
Consider this my application for the PhD position you had mentioned.

I have worked extensively on ROS, both during my course at FH Aachen and while working at Streetscooter GmbH, 
and am very thorough in my knowledge of its architecture. 
I have implemented various SLAM and VO algorithms/networks (ORBSLAM2, RGBD SLAM, vid2depth, sfmlearner) using OpenCV, G2O, Tensorflow etc. I also have experience working with Simulation/Game engines like Unity3D and Unreal Engine. All the above projects have made me quite fluent in Python and C++.


Furthermore, I do enjoy shell scripting and have good experience with the 
Linux system (Ubuntu, Fedora), Docker and SLURM scheduling.


Project links:


This was my Master Mechatronics project where i was part of a team 
that developed an omnidirectional AGV using ROS/Gazebo. I was primarily 
involved in the design of the position controller.

- Omnidirectional AGV with integrated low level controlller: 
\httplink{https://github.com/ajithkrishnan/agv}


I had modified these existing networks to run on custom training and 
test datasets at Streetscooter GmbH.

-  vid2depth: \httplink{https://gitlab.com/ajithkrishnan/models/tree/master/research/vid2depth}

- SfMLearner: \httplink{https://gitlab.com/ajithkrishnan/sfmlearner}


These are some of the Docker Images i had created to work on various projects.

- Docker Images: \httplink{https://github.com/ajithkrishnan/Docker} 




%\lipsum[1-2] % Dummy text
%\lipsum[4] % Dummy text

\makeletterclosing % Print letter signature

\newpage
%\end{comment}

%----------------------------------------------------------------------------------------
%	CURRICULUM VITAE
%----------------------------------------------------------------------------------------

\makecvtitle % Print the CV title

%----------------------------------------------------------------------------------------
%	WORK EXPERIENCE SECTION
%----------------------------------------------------------------------------------------

%\section{\textbf{\Large{E}\large{XPERIENCE}}}
\section{\scshape{Experience}}

%\subsection{\scshape{Work Experience}}
\cventry{\fontsize{9}{10.8}\selectfont{2018 - Now}}{Student Assistant}{Streetscooter GmbH}{Aachen, Germany}{}
{
\begin{itemize}
\item Developed and designed simulations of autonomous vehicles in Unity 3D and Unreal
    Engine
\item Implemented indoor/outdoor SLAM systems viz. RGBD SLAM, Elastic Fusion, 
LSD SLAM and ORB SLAM2 using Intel Realsense d415 RGBD camera    
\item Implemented various Unsupervised Learning Visual Odometry pipelines viz. Sfmlearner, vid2depth using Tensorflow
\newline{}
\end{itemize}}

%{Implemented various Unsupervised Learning Visual Odometry pipelines viz. Sfmlearner, vid2depth using Tensorflow\newline{}}


\cventry{\fontsize{9}{10.8}\selectfont{2014 - 2016}}{Mechanical Engineer}{JAL Engineering Services L.L.C}{Oman}{}{
\begin{itemize}
\item Designed an installed HVAC systems in commercial buildings in Muscat, Oman.
\newline{}
\end{itemize}}

%----------------------------------------------------------------------------------------
%	EDUCATION SECTION
%----------------------------------------------------------------------------------------

\section{\scshape{Education}}

\cventry{\fontsize{9}{10.8}\selectfont{2016 - Now}}{Masters(Msc.) in Mechatronics}{Fachhochschule Aachen}{NRW, Germany}{GPA:2.31}
{\begin{itemize}
\item Subjects:  Industrial Communication, ROS, Advanced Robotics and Autonomous Mobile Systems
\item Projects: 
\begin{itemize}
\item Development of Holonomic Automated Guided Vehicle - Implementing Autonomous Navigation and safety features using ROS. (ongoing)
\item PWFHRC - Developed Autonomous Navigation System to track Wildlife in Kruger National Park using ROS and Gazebo.(July 2017)
\item Arduino based Light Detection System - Developed a PID control system to manipulate the light intensity with a rolling shutter system(June 2017)
\item Arduino based CAN – Developed a network based on MCP2515 CAN module using Arduino Uno. (December 2016)
\newline{}
\end{itemize}
\end{itemize}  % Arguments not required can be left empty
}

\cventry{\fontsize{9}{10.8}\selectfont{2010 - 2014}}{Bachelors(B.Tech) in Mechanical Engineering}{College of Engineering}{Trivandrum, India}{GPA:7.56/10 (highest GPA:10.0)}
{\begin{itemize}
\item Subjects: Computer Programming and Numerical Methods, Computational Fluid Dynamics, Automobile Engineering
\item Main Project: Numerical Study of Blunt Cone in a Flow Field – Studied the Lift and Drag characteristics using ANYSYS FLUENT (September 2013 – May 2014)
\end{itemize}
}

%----------------------------------------------------------------------------------------
%	SOFTWARE SKILLS SECTION
%----------------------------------------------------------------------------------------

\section{\scshape{Software Skills}}

\cvitem{\fontsize{9}{10.8}\selectfont{Advanced}}{C++/C, Python, Matlab Simulink, Robotic Operating System(ROS), Gazebo, OpenCV}
\cvitem{\fontsize{9}{10.8}\selectfont{Intermediate}}{SolidWorks, AutoCAD\newpage}
%\cvitem{Basic}{Computer Hardware and Support}


%----------------------------------------------------------------------------------------
%	LANGUAGES SECTION
%----------------------------------------------------------------------------------------

\section{\scshape{Language Skills}}

\cvitemwithcomment{\fontsize{9}{10.8}\selectfont{Mother Tongue}}{\normalfont{Malayalam}}{}
\cvitemwithcomment{\fontsize{9}{10.8}\selectfont{Other Language(s)}}{\normalfont{English(TOEFL: 111/120)}}{}
\cvitemwithcomment{}{\begin{tabular}{|>{\centering\arraybackslash}p{3.5cm}|>{\centering\arraybackslash}p{2cm}|>{\centering\arraybackslash}p{2cm}|>{\centering\arraybackslash}p{2cm}|>{\centering\arraybackslash}p{2cm}|}
\hline
\normalfont{Language} & \normalfont{Listening} & \normalfont{Reading} & \normalfont{Speaking} & \normalfont{Writing}\\
\hline
\normalfont{Deutsch} & \normalfont{B1} & \normalfont{B1} & \normalfont{B1} & \normalfont{B1}\\
\hline
\end{tabular}
\newline{}\newline{}
\footnotesize{\normalfont{(*) Common European Framework of Reference(CEFR) Level}}}{}

%----------------------------------------------------------------------------------------
%	ADDITIONAL INFORMATION
%----------------------------------------------------------------------------------------

\section{\scshape{Additional Information}}

\cventry{\fontsize{9}{10.8}\selectfont{2018 - Now}}{Member}{Foodsharing}{Aachen, Germany}{}{Food sharing is a 2012 initiative against food waste that ßaves"food that would otherwise
be thrown away. Over 200,000 registered users in Germany/Austria/Switzerland and
over 25,000 volunteers, so-called food savers, make this initiative an international
movement.}
\cventry{\fontsize{9}{10.8}\selectfont{2013 - 2014}}{Public Relations Head}{Make A Difference}{India}{}{Make A Difference is a platform that empowers youth to become change leaders who make positive social impact and create self-sustaining communities by educating nearly 4000 children across 23 cities in India.}
\cventry{\fontsize{9}{10.8}\selectfont{2013 - 2013}}{Event Organiser}{TEDxCET}{India}{}{The TEDx Program is designed to help communities, organizations and individuals to spark conversation and connection through local TED-like experiences.}

\begin{comment}
\cvitem{\fontsize{9}{10.8}\selectfont{2008-Present}}{Member of Toastmasters International}
\cvitem{\fontsize{9}{10.8}\selectfont{03.12.2011}}{Attended ‘Haptic RoboArm’ workshop conducted by IEEE SB-TKMIT, India.}

%----------------------------------------------------------------------------------------
%	INTERESTS SECTION
%----------------------------------------------------------------------------------------

\section{Interests}

\renewcommand{\listitemsymbol}{-~} % Changes the symbol used for lists

\cvlistdoubleitem{Piano}{Chess}
\cvlistdoubleitem{Cooking}{Dancing}
\cvlistitem{Running}

\end{comment}
%----------------------------------------------------------------------------------------

\end{document}