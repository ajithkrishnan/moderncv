%%%%%%%%%%%%%%%%%%%%%%%%%%%%%%%%%%%%%%%%%
% "ModernCV" CV and Cover Letter
% LaTeX Template
% Version 1.3 (29/10/16)
%
% This template has been downloaded from:
% http://www.LaTeXTemplates.com
%
% Original author: 
% Xavier Danaux (xdanaux@gmail.com) with modifications by:
% Vel (vel@latextemplates.com)
%
% License:
% CC BY-NC-SA 3.0 (http://creativecommons.org/licenses/by-nc-sa/3.0/)
%
% Important note:
% This template requires the moderncv.cls and .sty files to be in the same 
% directory as this .tex file. These files provide the resume style and themes 
% used for structuring the document.
%
%%%%%%%%%%%%%%%%%%%%%%%%%%%%%%%%%%%%%%%%%

%----------------------------------------------------------------------------------------
%	PACKAGES AND OTHER DOCUMENT CONFIGURATIONS
%----------------------------------------------------------------------------------------
 
\documentclass[11pt,a4paper,roman]{moderncv} % Font sizes: 10, 11, or 12; paper sizes: a4paper, letterpaper, a5paper, legalpaper, executivepaper or landscape; font families: sans or roman

\moderncvstyle{classic} % CV theme - options include: 'casual' (default), 'classic', 'oldstyle' and 'banking'
\moderncvcolor{blue} % CV color - options include: 'blue' (default), 'orange', 'green', 'red', 'purple', 'grey' and 'black'

\usepackage{lipsum} % Used for inserting dummy 'Lorem ipsum' text into the template

\usepackage[scale=0.825]{geometry} % Reduce document margins
%\setlength{\hintscolumnwidth}{3cm} % Uncomment to change the width of the dates column
%\setlength{\makecvtitlenamewidth}{10cm} % For the 'classic' style, uncomment to adjust the width of the space allocated to your name

% to comment out blocks 
\usepackage{verbatim} 

% to center align and specify column width in tables
\usepackage{array}

% german Language, new writing rules
\usepackage[T1]{fontenc}
\usepackage[utf8]{inputenc}
\usepackage[ngerman]{babel}  %<========== Language German, new orthography

\usepackage{fontawesome}
\usepackage{etoolbox}

\usepackage{csquotes}
\usepackage{ragged2e}
%\usepackage{hyperref}
\usepackage{textcase}

%----------------------------------------------------------------------------------------
%	NAME AND CONTACT INFORMATION SECTION
%----------------------------------------------------------------------------------------
\patchcmd{\makecvtitle}{[b]}{[t]}{}{}
%\patchcmd{\makecvtitle}{[b]}{[t]}{}{}

\setlength{\makecvheadnamewidth}{13cm}
\renewcommand*{\namefont}{\fontsize{31}{32}\mdseries\upshape}

%\firstname{\scshape{Ajith}} % Your first name
%\familyname{\scshape{Krishnan}} % Your last name

\name{\scshape{Ajith}}{\scshape{Krishnan}}
%\name{Ajith}{Krishnan}

% All information in this block is optional, comment out any lines you don't need
\title{\Large{Curriculum Vitae}}
%\address{Scheibenstraße 14}{Aachen, Deutschland 52070}
\mobile{+49 15758660972}
\social[linkedin]{ajith-krishnan}
%\social[github]{ajithkrishnan} 
\extrainfo{
    %\gitlabsocialsymbol\httplink{gitlab.com/ajithkrishnan} \\
    %\social[gitlab]{ajithkrishnan} \\
    DOB: 04.01.1993  \\
    Nationality: India \\ 
    \emailsymbol\emaillink{ajith.krishnan@alumni.fh-aachen.de} 
    }
%\email{ajith.krishnan@alumni.fh-aachen.de}
% \homepage{www.github.com/ajithkrishnan}{staff.org.edu/$\sim$jsmith} % The first argument is the url for the clickable link, the second argument is the url displayed in the template - this allows special characters to be displayed such as the tilde in this example

%\social[gitlab]{ajithkrishnan} 

%\extrainfo{\httplink{github.com/ajithkrishnan}}
% The first bracket is the picture height, the second is the thickness of the frame around the picture (0pt for no frame)
\photo[89pt][0.3pt]{../pictures/bild.jpg} 
%\photo[89pt][0.3pt]{../examples/picture.jpg} 
%\quote{"A witty and playful quotation" - John Smith}

%----------------------------------------------------------------------------------------

\begin{document}


%----------------------------------------------------------------------------------------
%	APPLICATION SUMMARY SECTION
%----------------------------------------------------------------------------------------

\begin{comment}

\begingroup
    \Huge{\textbf{Application Summary}}
    \vspace{30pt}

    \Large{\textbf{Their Profile}}
    \begin{itemize}
        \item RWTH Aachen
        \item Prof. Dr.-Ing. Jakob Andert
        \item Research and development - Mechanical, Electronic, IT - elements in modern vehicle drives
        
    \end{itemize}

    \vspace{15pt}
    \Large{\textbf{My Profile}}
    \begin{itemize}
        \item Bachelors degree - in Computer Science or related fields
        \item Good knowledge - data communication, databases, embedded systems, computer graphics, computer Vision
        \item Good experience - linux systems - implement machine learning algorithms
        \item Sound knowledge - C, C++, Python, Java
        \item Basic knowledge - mechanical
        \item Highly independent - structured way of working
        \item {\color{red} Fluent in English and German} (spoken and written)
        
    \end{itemize}

    \vspace{15pt}
    \Large{\textbf{Tasks}}
    \begin{itemize}
        \item Manage Linux-based traffic simulations with tool Sumo
        \item Planning, Execution and Analysis - Traffic-simulation - Machine learning
        \item 
    \end{itemize}

    \vspace{15pt}
    \Large{\textbf{Offer}}
    \begin{itemize}
        \item EG 10 TV-L
    \end{itemize}

\endgroup

\newpage
 
\end{comment}



%----------------------------------------------------------------------------------------
%	COVER LETTER
%----------------------------------------------------------------------------------------

% To remove the cover letter, comment out this entire block

\clearpage

%----------------------------------------------------------------------------------------
%	EDIT BEFORE SENDING
%----------------------------------------------------------------------------------------
\newcommand{\tofill}[1]{\textbf{{\color{red}#1}}\hspace{1mm}}

\opening{Dear \tofill{recipient},} % Opening greeting
\newcommand{\institute}{\tofill{institute}\hspace{1mm}}
\newcommand{\position}{\tofill{position}\hspace{1mm}}
\newcommand{\applicationnr}{\tofill{application nr.}}
\recipient{\position (Nr. \applicationnr)}{\institute} % Letter recipient
\newcommand{\instidescr}{\tofill{something positive about institute}}
\newcommand{\projdescr}{\tofill{project description}}

%----------------------------------------------------------------------------------------
%	EDIT BEFORE SENDING
%----------------------------------------------------------------------------------------


\date{\today} % Letter date
\closing{Sincerely yours,} % Closing phrase
\enclosure[Attached]{curriculum vit\ae{}} % List of enclosed documents

%\begin{comment}

\makelettertitle % Print letter title

\justify

My name is Ajith Krishnan and while I have a strong background in \tofill{background} and experience with \tofill{background}  
it is the opportunity to work as \position at \institute that excites me. \instidescr. 
I have always wanted to work on projects on \projdescr and have a passion for guiding projects to completion.

I have completed my masters in Mechatronics from FH Aachen. During my course I have implemented and trained various model-based and 
deep-learning based computer-vision pipelines to improve the localization accuracy of autonomous vehicles. I am well versed with Git and Docker, 
having worked with these systems on a daily basis as Student Assistant at Streetscooter GmbH. Additionally I have very good experience with the Robotic Operative System (ROS) software stack 
and the Linux OS. Developing these softwares in Python and C++ environments has led to a thorough understanding of object oriented design.
    
My knowledge in these fields were basic when I began my work with Streetscooter. However over the course of my two year contract, 
I have realized that I can start from a blank slate and adapt my knowledge in new fields and deliver within the required timespan.
What I can bring to your team is good passion for research and the ability to go from ideation to developing the necessary mathematical models 
and finally analyzing their performances on evaluation data. I have attached a letter of recommendation from my supervisor that vouches for this.

This position seeks the expertise and interests that I have acquired over the years and I believe that I will be a good addition to your team. 
I request you to provide me a chance to showcase my drive.

\vspace{20pt}
\makeletterclosing % Print letter signature



%\end{comment}


\newpage
%\end{comment}

%\begin{comment}
%
%----------------------------------------------------------------------------------------
%	CURRICULUM VITAE
%----------------------------------------------------------------------------------------

\makecvtitle % Print the CV title

%----------------------------------------------------------------------------------------
%	EDUCATION SECTION
%----------------------------------------------------------------------------------------

\section{\scshape{\huge Education}}

\vspace{10pt}

%\cventry{\fontsize{9}{10.8}\selectfont{September 2016 - March 2020}}{Masters(Msc.) in Mechatronics}{Fachhochschule Aachen}{Germany}{GPA:2.0}
%\cventry{\fontsize{9}{10.8}\selectfont{September 2016 - March 2020}}{\large Master Mechatronics}{\large \textbf{FH Aachen}}{\large Germany}{}
\cventry{\fontsize{9}{10.8}\selectfont{September 2016 - March 2020}}{\large Master Mechatronics}{\large \textbf{FH Aachen}}{\large Germany}{}
{
\textbf{CGPA: 2.0/4.0}
\vspace{5pt}
\begin{itemize}
\item Focus: Advanced Mathematics, Advanced Robotics and Autonomous Mobile Systems, ROS
%ROS, Advanced Fabrication Technology 
%Systems Engineering,
%\item Master Project: Development of a Control System for a Holonomic Automated Guided Vehicle (AGV)
%Developed a two-wheeled balance bot
%Developed Autonomous Navigation System to track Wildlife in Kruger National Park using ROS and Gazebo.
\item Thesis: Visual-Wheel Odometry based Localization of Logistic Transporters
%\item Additional: Conducted and tutored 50 students to develop an autonomous rover using the ROS framework during ROS Summer School, 2017 
\end{itemize}
}

\vspace{10pt}

%\cventry{\fontsize{9}{10.8}\selectfont{May 2010 - April 2014}}{Bachelors(B.Tech) in Mechanical Engineering}{College of Engineering}{Trivandrum, India}{GPA:7.56/10 (highest GPA:10.0)}
\cventry{\fontsize{9}{10.8}\selectfont{May 2010 - April 2014}}{\large Bachelor Mechanical}{\large \textbf{College of Engineering Trivandrum}}{\large India}{}
{
\textbf{CGPA: 7.56/10.0}
\vspace{5pt}
\begin{itemize}
\item Focus: Computer Programming and Numerical Methods, Computational Fluid Dynamics, Engineering Mathematics, Automobile Engineering
\item Thesis: Numerical Study of Blunt Cone in a Flow Field
\end{itemize}
}



%----------------------------------------------------------------------------------------
%	WORK EXPERIENCE SECTION
%----------------------------------------------------------------------------------------

%\section{\textbf{\Large{E}\large{XPERIENCE}}}
\vspace{20pt}
\section{\scshape{\huge Experience}}
\vspace{10pt}
%\subsection{\scshape{Work Experience}}
\cventry{\fontsize{9}{10.8}\selectfont{April 2018 - March 2020}}{\large Student Assistant}{\large \textbf{Streetscooter GmbH}}{\large Aachen, Germany}{}
{
\begin{itemize}
%\item Designed and implemented learning-based non-linear Odometry model using wheel- and steering-encoder signals
\item Trained a neural network to learn non-linearities in wheel rotations of a vehicle which lead to improved wheel odometry predictions than 
existing model-based approaches
\item Implemented and tested visual odometry and SLAM pipelines on a prototype vehicle to complement the existing localization subsystem
%\item Implementation of various driving functionalities in simulation environments like Gazebo, Unity3D, UnrealEngine
\item Built a bridge between LGSVL driving simulation (Unity3D) and Autonomous Driving software and integrated LG G920 steering wheel 
for client/shareholder demonstrations
%\item Implemented state of the art Deep Learning based monocular visual odometry systems using Tensorflow/Keras
%\item Built and deployed Docker containers on SLURM based Linux server to implement above mentioned projects
\item Built and deployed Docker containers on SLURM scheduling based Linux servers in order to improve computation time of simulations and deep neural networks
\end{itemize}
}

%{Implemented various Unsupervised Learning Visual Odometry pipelines viz. Sfmlearner, vid2depth using Tensorflow\newline{}}
\vspace{10pt}

\cventry{\fontsize{9}{10.8}\selectfont{July 2014 - February 2016}}{\large Mechanical Engineer}{\large \textbf{JAL Engineering Services L.L.C}}{\large Muscat, Oman}{}{
\begin{itemize}
%\item Designed and installed Potable water pipeline of Sea Water Intake Pumping Station(SWIPS) in Sohar, Oman
\item Designed, installed and tested an Automatic Car Park Guidance System for Muscat Avenues Mall which significantly reduced traffic congestion
\item Designed, installed and tested HVAC systems in numerous commercial buildings in Muscat, Oman
\end{itemize}
}

\newpage

%----------------------------------------------------------------------------------------
%	PROJECTS AND SEMINARS SECTION
%----------------------------------------------------------------------------------------
\vspace{20pt}
\section{\scshape{\huge Projects and Seminars}}
\vspace{10pt}
\cventry{\fontsize{9}{10.8}\selectfont{August 2017 - March 2018}}{\large Two-wheeled balance bot}{\large \textbf{FH Aachen}}{}{}
{Designed a remote controlled two-wheeled balance bot using Systems Engineering techniques (SysML, UML) with the aim of learning iterative design and product development}
\cventry{\fontsize{9}{10.8}\selectfont{August 2017 - March 2018}}{\large Autonomous Navigation System for wildlife tracking, \textbf{FH Aachen}}{}{}{}
{Simulated a SLAM-based navigation(ROS/Gazebo) using a UGV to track endangered gazelle herds on uneven terrains in Kruger National Park, South Africa}
\cventry{\fontsize{9}{10.8}\selectfont{August 2017}}{\large ROS Summer School}{\large \textbf{FH Aachen}}{}{}
{Conducted and tutored 30 students to develop an autonomous rover using the ROS framework during ROS Summer School, 2017}
\cventry{\fontsize{9}{10.8}\selectfont{August 2018 - May 2019}}{\large Control System for a Holonomic Automated Guided Vehicle (AGV)}{\large \textbf{FH Aachen}}{}{}
{Developed a lower level PID control system for an industrial holonomic AGV for risk management in case of software-stack failure}

%----------------------------------------------------------------------------------------
%	SOFTWARE SKILLS SECTION
%----------------------------------------------------------------------------------------
\vspace{20pt}
\section{\scshape{\huge Software Skills}}
\vspace{10pt}

\subsection{\scshape{Programming Languages}}

\cvitem{\fontsize{9}{10.8}\selectfont{Advanced}}{Python}
\cvitem{\fontsize{9}{10.8}\selectfont{Intermediate}}{C++}

\subsection{\scshape{Frameworks/Softwares}}
\cvitem{\fontsize{9}{10.8}\selectfont{Advanced}}{Robotic Operating Software(ROS), Numpy, Pandas, Scipy}
\cvitem{\fontsize{9}{10.8}\selectfont{Intermediate}}{Tensorflow, Keras, OpenCV, PointCloud}

\subsection{\scshape{Simulation/Modelling}}
\cvitem{\fontsize{9}{10.8}\selectfont{Intermediate}}{Unity3D, Unreal Engine (UE4), Gazebo}

\subsection{\scshape{DevOps}}
\cvitem{\fontsize{9}{10.8}\selectfont{Advanced}}{Linux OS/Bash}
\cvitem{\fontsize{9}{10.8}\selectfont{Intermediate}}{Git, Docker}

\subsection{\scshape{Processes}}
\cvitem{\fontsize{9}{10.8}\selectfont{Basic}}{Agile/Scrum, CI/CD pipeline}

%\cvitem{\fontsize{9}{10.8}\selectfont{Advanced}}{Python, C++, Robotic Operating System(ROS), Linux OS}
%\cvitem{\fontsize{9}{10.8}\selectfont{Intermediate}}{Matlab/SIMULINK, Unity3D, Unreal Engine\newpage}
%\cvitem{Basic}{Computer Hardware and Support}


%----------------------------------------------------------------------------------------
%	LANGUAGES SECTION
%----------------------------------------------------------------------------------------
\vspace{20pt}
\section{\scshape{\huge Language Skills}}
\vspace{10pt}

%BULLETED FORMAT
\cvitemwithcomment{\fontsize{9}{10.8}\selectfont{Business Fluent}}{\normalfont{English (TOEFL: 111/120)}}{}
\cvitemwithcomment{\fontsize{9}{10.8}\selectfont{Fluent}}{\normalfont{German (B1*)}}{}
\cvitemwithcomment{\fontsize{9}{10.8}\selectfont{Native}}{\normalfont{Malayalam}}{}
\begingroup
    \footnotesize{\normalfont{(*) Common European Framework of Reference(CEFR) Level}}
\endgroup

% TABULAR FORMAT
\begin{comment}

\cvitemwithcomment{\fontsize{9}{10.8}\selectfont{Mother Tongue}}{\normalfont{Malayalam}}{}
%\cvitemwithcomment{\fontsize{9}{10.8}\selectfont{Other Language(s)}}{\normalfont{English(TOEFL: 111/120)}}{}
\cvitemwithcomment{\fontsize{9}{10.8}\selectfont{Other Language(s)}}{\begin{tabular}{|>{\centering\arraybackslash}p{3.5cm}|>{\centering\arraybackslash}p{2cm}|>{\centering\arraybackslash}p{2cm}|>{\centering\arraybackslash}p{2cm}|>{\centering\arraybackslash}p{2cm}|}
\hline
\normalfont{Language} & \normalfont{Listening} & \normalfont{Reading} & \normalfont{Speaking} & \normalfont{Writing}\\
\hline
\normalfont{Deutsch} & \normalfont{B1} & \normalfont{B1} & \normalfont{B1} & \normalfont{B1}\\
\hline
\normalfont{English} & \normalfont{C2} & \normalfont{C2} & \normalfont{C2} & \normalfont{C2}\\
\hline
\end{tabular}
%\newline{}\newline{}
%\footnotesize{\normalfont{(*) Common European Framework of Reference(CEFR) Level}}
}{}
\hspace{25mm} 
\begingroup
    \footnotesize{\normalfont{(*) Common European Framework of Reference(CEFR) Level}}
\endgroup

\end{comment}

%----------------------------------------------------------------------------------------
%	VOLUNTEER EXPERIENCE
%----------------------------------------------------------------------------------------
\begin{comment}
\section{\scshape{\huge Volunteer Experience}}
\vspace{10pt}

\cventry{\fontsize{9}{10.8}\selectfont{June 2018 - Present}}{Member}{Foodsharing}{Aachen, Germany}{}{Food sharing is a 2012 initiative 
against food waste that saves food that would otherwise
be thrown away. Over 200,000 registered users in Germany/Austria/Switzerland and
over 25,000 volunteers, so-called food savers, make this initiative an international
movement.}

\vspace{10pt}

\cventry{\fontsize{9}{10.8}\selectfont{May 2013 - April 2014}}{Public Relations Head}{Make A Difference}{India}{}{Make A Difference is
 a platform that empowers youth to become change leaders who make positive social impact and create self-sustaining communities
  by educating nearly 4000 children across 23 cities in India.}

\end{comment}


%\cventry{\fontsize{9}{10.8}\selectfont{June 2013 - 2013}}{Event Organiser}{TEDxCET}{India}{}{The TEDx Program is designed to help communities, organizations and individuals to spark conversation and connection through local TED-like experiences.}

% \end{comment}

\begin{comment}
\cvitem{\fontsize{9}{10.8}\selectfont{2008-Present}}{Member of Toastmasters International}
\cvitem{\fontsize{9}{10.8}\selectfont{03.12.2011}}{Attended ‘Haptic RoboArm’ workshop conducted by IEEE SB-TKMIT, India.}

%----------------------------------------------------------------------------------------
%	INTERESTS SECTION
%----------------------------------------------------------------------------------------

\section{Interests}

\renewcommand{\listitemsymbol}{-~} % Changes the symbol used for lists

\cvlistdoubleitem{Piano}{Chess}
\cvlistdoubleitem{Cooking}{Dancing}
\cvlistitem{Running}

\end{comment}
%----------------------------------------------------------------------------------------



\end{document}
